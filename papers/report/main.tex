\documentclass{article}

\usepackage[utf8]{inputenc}

\usepackage[margin=0.8in]{geometry}
\usepackage{parskip}
\usepackage[english]{babel}

\usepackage{../styles/hocki-klocki}

\theoremstyle{definition}
\newtheorem{definition}{Definition}[section]
\theoremstyle{definition}
\newtheorem{notation}{Notation}[section]
\theoremstyle{definition}
\newtheorem{theorem}{Theorem}[section]
\theoremstyle{definition}
\newtheorem{remark}{Remark}[section]
\theoremstyle{definition}
\newtheorem{corollary}{Corollary}[section]

\title{Hocki Klocki}
\author{Michał Horodecki \\ Jakub Oskwarek}
\date{January 2025}

\begin{document}

\maketitle

\section{Smol Bebi Version}

The most basic version without even dependent dimensions.

\subsection{Constraints}

In the most basic we have the following constraints:
\begin{itemize}
    \item \( \dimension \in \dimSetVarIn \) -- dimension \( \dimension \) has to be in dim set assigned to \( \dimSetVarIn \)
    \item \( \dimension \notin \dimSetVarIn \) -- dimension \( \dimension \) cannot be in dim set assigned to \( \dimSetVarIn \)
    \item \( \dimension \in \union_{i \in I} \dimSetVarInI{i} \) -- dimension \( \dimension \) has to be in union of dim sets assigned to all \( \dimSetVarInI{i} \)
    \item \( \dimSetVarOut \inducedBy \union_{i \in I} \dimSetVarInI{i} \setminus \dimSet \)
    \item \( \dimSetVarIn = \dimSetVarOut \) -- dim sets assigned to \( \dimSetVarIn \) and \( \dimSetVarOut \) have to be equal
\end{itemize}

\subsection{Primitive blocks}

We introduce three primitive block schemas that introduce the above constraints:
\begin{enumerate}
    \item Introduction of dimension \( \dimension \),  with the following constraints:
        \begin{itemize}
            \item \( \dimSetVarOut \supseteq \dimSetVarIn \setminus \set{\dimension} \)
            \item \( \dimension \notin \dimSetVarIn \)
            \item \( \dimension \in \dimSetVarOut \)
        \end{itemize}
    \item Removal of dimension \( a \),  with the following constraints:
        \begin{itemize}
            \item \( Y \supseteq X \setminus \set{a} \)
            \item \( a \in X \)
            \item \( a \notin Y \)
        \end{itemize}
    \item Union of dimensions
        \begin{itemize}
            \item \( Y \supseteq \bigcup_{i \in I} X_i \)
        \end{itemize}
\end{enumerate}

\subsection{Semantics}
\begin{notation}
    The notation ``\(\pm\set{\delta\ldots}\) block'' is shorthand for ``any \(\pm\Delta\) block such that \(\delta \in \Delta\)''.
\end{notation}

\begin{definition}
    An expanded program is a block constructed by recursively expanding nested blocks down to primitive blocks.
\end{definition}

\begin{definition}
    In an expanded program, a \(\delta\)-propagating path is a~path from a~\(+\set{\delta\ldots}\) block and without a~\(-\set{\delta\ldots}\) block.
\end{definition}

\begin{definition}
    Let \(\mathcal{V}\) be the vertices of an expanded program \(P\). A function \(f : \mathcal{V} \function \powerset\pars{\mathcal{D}}\) is called \(P\)-valid if both of the following conditions are met:
    \begin{enumerate}
        \item every primitive block has its inherent constraints satisfied
        \item for every \(X \in \mathcal{V}\) and \(\delta \in \mathcal{D}\) such that \(\delta \in f\pars{X}\) there exists a \(\delta\)-propagating path ending on \( X \).
    \end{enumerate}
\end{definition}

\begin{remark}
    The definition above allows for stubbing program inputs by insertion of a \(0\)-ary union (whose output is necessarily an empty set of dimensions) followed by a~\(+\Delta\) block, where \(\Delta\) is the desired input set of dimensions.
\end{remark}\subsection{Type Reconstruction Algorithm}

Given a schema \( s \) that consists of blocks \( s_1, \dots, s_n \), to type \( s \):
\begin{enumerate}
    \item Recursively find types of schemata \( s_1, \dots, s_n \).
    \item For every dimension \( a \) occurring in the constraints of \( s_1, \dots, s_n \) define relation \( X \Rightarrow_a Y \) iff exists a constraint \( Y \supseteq \bigcup_{i \in I} X_i \setminus A_i \) where \( X = X_j \) and \( a \notin A_j \) for some \( j \).
    \item Compute transitive and reflective closure of all relations \( \Rightarrow_a \)
    \item Propagate constraints \( a \in X \) down, i.e.:
    \[
        \infer{a \in Y}{a \in X & X \Rightarrow_a Y}
    \]
    
    \item Propagate constraints \( a \notin Y \) up, i.e.:
    \[
        \infer{a \notin X}{a \notin Y & X \Rightarrow_a Y}
    \]
    
    \item Propagate constraints \( a \in \bigcup_{i \in I} X_i \) up, i.e.:
    \[
        \infer{a \in \pars{Y_j \cup \bigcup_{i \in I} X_i}}{a \in \bigcup_{i \in I} X_i & X_j \Rightarrow_a Y_j}    
    \]
    
    \item Narrow the constraints down to those regarding interface vertices of \( s \).
\end{enumerate}


\subsection{Proof of correctness}

\begin{theorem}[Soundness]
    Let \( C_T(s) \) be the set of constraints returned by the above algorithm. Then any \(C_T\pars{s}\)-consistent function \( f : I(s) \rightarrow \mathcal{P}(\mathcal{D}) \) can be extended to an \(s\)-valid function \( \widehat f : V(s) \rightarrow \mathcal{P}(\mathcal{D}) \).
\end{theorem}
\begin{proof}
by sorry
\end{proof}

\begin{theorem}[Completeness]
    Every \(s\)-valid \(f\) is \(C_T\pars{s}\)-consistent.
\end{theorem}
\begin{proof}
TBA?
\end{proof}

\begin{corollary}
    The least \(C_T\pars{s}\)-consistent \(f\) can be extended to the least \(s\)-valid \(\widehat{f}\).
\end{corollary}
\begin{proof}
    Let \( \widehat{f} \) be the least \(s\)-valid function. It is \(C_T\pars{s}\)-consistent by completeness. Let \( f = \widehat{f}_{\mid I(s)} \). We will show that it is the least \(C_T\pars{s}\)-consistent function.
    
    Let \( g \) be \(C_T(s) \)-consistent, and let \( \widehat{g} \) be the extension achieved by soundness.
    
    We have \( \widehat{f} \leq \widehat{g} \), therefore \( f = \widehat{f}_{\mid I(s)} \leq \widehat{g}_{\mid I(s)} = g \)
    
\end{proof}

\end{document}
